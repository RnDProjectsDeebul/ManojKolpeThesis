%!TEX root = ../report.tex

\begin{document}
    \begin{abstract}

	Semantic segmentation is a technique used in computer vision to assign labels to each pixel in an image or video. It is often used in safety-critical applications, where the efficiency of the segmentation model is important. In video segmentation, the task is typically performed on key frames or all frames in a sequence. Overlapping information between consecutive frames can be used to improve the performance of semantic segmentation through temporal fusion. In this study, two separate temporal fusion methods using Gaussian Process and Long Short Term Memory (LSTM) were developed and applied to the latent space embedding of an encoder-decoder model, and their results were compared to a baseline model. The LSTM model performed well on the scannet and vkitti datasets, achieving mIoU scores of 0.65 and 0.72, respectively, and accuracy scores of 0.85 and 0.94, respectively.
	
%	This research examined the use of temporal fusion to improve the performance of semantic segmentation, a technique that assigns labels to each pixel in an image or video. In safety-critical applications, the efficiency of the model is a key consideration. Video segmentation, where the task is typically performed on key frames or all frames in a video sequence, can benefit from the incorporation of overlapping information between consecutive frames through temporal fusion. In this study, two temporal fusion methods using Gaussian Process and Long Short Term Memory (LSTM) were developed and applied to the latent space embedding of an encoder-decoder model, and their results were compared to a baseline model. The LSTM model performed well on the scannet and vkitti datasets, achieving mIoU scores of 0.65 and 0.72, respectively, and accuracy scores of 0.85 and 0.94, respectively.
        
        
%        Semantic segmentation is a technique used in computer vision to assign labels to each pixel in an image or video. It is often used in safety-critical applications, where the efficiency of the segmentation model is important. In video segmentation, the task is typically performed on key frames or all frames in a sequence. Overlapping information between consecutive frames can be used to improve the performance of semantic segmentation through temporal fusion. In this study, a temporal fusion method was developed using a combination of Gaussian Process and Long Short Term Memory (LSTM) to model the temporal data in the latent space embedding of an encoder-decoder model. The performance of this method was compared to a baseline Unet model and a Unet model with temporal fusion using Gaussian Process. The LSTM model performed well on the scannet and vkitti datasets, achieving mIoU scores of 0.65 and 0.72, respectively, and accuracy scores of 0.85 and 0.94, respectively.
 

      
    \end{abstract}
\end{document}
